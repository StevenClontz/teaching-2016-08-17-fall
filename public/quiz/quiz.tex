\documentclass[12pt]{exam}

\newcommand{\ds}{\ensuremath{\displaystyle}}

\usepackage{amsmath,amsfonts, amsthm}
\usepackage{multicol}
\usepackage{multirow}
\usepackage{harpoon}
\renewcommand{\arraystretch}{1.5}

\newcommand{\harpvec}[1]{\overrightharp{\ensuremath{\mathbf{#1}}}}
\newcommand{\vect}[1]{\ensuremath{\mathbf{#1}}}
\newcommand{\<}{\langle}
\renewcommand{\>}{\rangle}
\newcommand{\p}{\partial}

% ref: http://pgfplots.sourceforge.net/gallery.html
% ref: http://tex.stackexchange.com/a/74575/79754
\usepackage{pgfplots}% This uses tikz
\pgfplotsset{compat=newest}% use newest version
\tikzset{LineStyle/.style={smooth, ultra thick, samples=400}}

% \printanswers

\begin{document}

\begin{center}
\fbox{\fbox{\parbox{5.5in}{\centering
MA 126 | Fall 2016 | Prof. Clontz | Quiz
}}}
\end{center}
\vspace{0.1in}
\makebox[\textwidth]{
  Name:\enspace\hrulefill\hrulefill\hrulefill
}

\vspace{12pt}

\textbf{Choose D for ``None of these''}

\begin{questions}\setcounter{question}{28}

\question
Find \(\ds\lim_{n\to\infty}\frac{n!\cos n}{(n+1)!}\).
\begin{choices}
\item \(1\)
\item \(0\)
\item \(\pi/2\)
\end{choices}

\question
Find \(\ds\lim_{n\to\infty}\frac{(3+n)^n}{n^n}\).
\begin{choices}
\item
  \(1\)
\item
  \(0\)
\item
  \(e^3\)
\end{choices}

\question
Which of these statements seems most appropriate for describing the
sequence whose initial terms are
\(\<\frac{1}{4},-\frac{1}{6},\frac{1}{8},-\frac{1}{10},\frac{1}{12},\dots\>\)?
\begin{choices}
\item The sequence is bounded and monotonic, so it converges by
      the Monotonic Sequence Theorem.
\item The sequence is not monotonic and not bounded, so it diverges by
      the Monotonic Sequence Theorem.
\item The sequence is bounded, but not monotonic, so the Monotonic Sequence
      Theorem doesn't apply. However, it does
      appear to converge to \(0\) anyway.
\end{choices}





\end{questions}

\end{document}
