\documentclass[12pt]{exam}

\newcommand{\ds}{\ensuremath{\displaystyle}}

\usepackage{amsmath,amsfonts, amsthm}
\usepackage{multicol}
\usepackage{multirow}
\usepackage{harpoon}
\renewcommand{\arraystretch}{1.5}

\newcommand{\harpvec}[1]{\overrightharp{\ensuremath{\mathbf{#1}}}}
\newcommand{\vect}[1]{\ensuremath{\mathbf{#1}}}
\newcommand{\<}{(}
\renewcommand{\>}{)}
\newcommand{\p}{\partial}

% ref: http://pgfplots.sourceforge.net/gallery.html
% ref: http://tex.stackexchange.com/a/74575/79754
\usepackage{pgfplots}% This uses tikz
\pgfplotsset{compat=newest}% use newest version
\tikzset{LineStyle/.style={smooth, ultra thick, samples=400}}

% \printanswers

\begin{document}

\begin{center}
\fbox{\fbox{\parbox{5.5in}{\centering
MA 126 | Fall 2016 | Prof. Clontz | Quiz
}}}
\end{center}
\vspace{0.1in}
\makebox[\textwidth]{
  Name:\enspace\hrulefill\hrulefill\hrulefill
}

\vspace{12pt}

\textbf{Choose D for ``None of these''}

\begin{questions}\setcounter{question}{20}

\question
Which of these polar coordinates gives the point \((-\sqrt3,1)\)?
\begin{choices}
\item \(p(\sqrt2,3\pi/4)\)
\item \(p(\sqrt3,\pi/3)\)
\item \(p(2,5\pi/6)\)
\end{choices}

\question
Convert the circle \(x^2+(y-4)^2=16\) into a polar equation.
\begin{choices}
\item \(r=16\)
\item \(r=8\sin\theta\)
\item \(r=12\cos^2\theta-4\sin^2\theta\)
\end{choices}

\question
Which of these equations gives the curve drawn on the board?
\begin{choices}
\item \(r=3+3\cos\theta\)
\item \(r=3-3\sin\theta\)
\item \(r=3\tan\theta\)
\end{choices}




\end{questions}

\end{document}
