\documentclass[12pt]{exam}

\newcommand{\ds}{\ensuremath{\displaystyle}}

\usepackage{amsmath,amsfonts, amsthm}
\usepackage{multicol}
\usepackage{multirow}
\usepackage{harpoon}
\renewcommand{\arraystretch}{1.5}

\newcommand{\harpvec}[1]{\overrightharp{\ensuremath{\mathbf{#1}}}}
\newcommand{\vect}[1]{\ensuremath{\mathbf{#1}}}
\newcommand{\<}{\langle}
\renewcommand{\>}{\rangle}
\newcommand{\p}{\partial}

% ref: http://pgfplots.sourceforge.net/gallery.html
% ref: http://tex.stackexchange.com/a/74575/79754
\usepackage{pgfplots}% This uses tikz
\pgfplotsset{compat=newest}% use newest version
\tikzset{LineStyle/.style={smooth, ultra thick, samples=400}}

% \printanswers

\begin{document}

\begin{center}
\fbox{\fbox{\parbox{5.5in}{\centering
MA 126 | Fall 2016 | Prof. Clontz | Quiz
}}}
\end{center}
\vspace{0.1in}
\makebox[\textwidth]{
  Name:\enspace\hrulefill\hrulefill\hrulefill
}

\vspace{12pt}

\textbf{Choose D for ``None of these''}

\begin{questions}\setcounter{question}{25}

\question
What are the first five terms of the sequence \(\<r_n\>_{n=1}^\infty\)
defined explicitly by \(r_n=\frac{n+2}{3+n^2}\)?
\begin{choices}
\item
  \(\<\frac{3}{4},\frac{4}{7},\frac{5}{12},\frac{6}{19},\frac{1}{4},\dots\>\)
\item
  \(\<\frac{2}{7},\frac{1}{2},\frac{4}{9},0,\frac{5}{17},\dots\>\)
\item
  \(\<0,\frac{3}{5},\frac{5}{18},\frac{8}{27},\frac{9}{61},\dots\>\)
\end{choices}

\question
What are the first five terms of the sequence \(\<w_n\>_{n=0}^\infty\)
defined recursively by \(w_0=1\), \(w_1=2\), \(w_{n+2}=2w_n+w_{n+1}\)?
\begin{choices}
\item
  \(\<1,2,5,10,17,\dots\>\)
\item
  \(\<1,2,3,5,9,\dots\>\)
\item
  \(\<1,2,4,8,16,\dots\>\)
\end{choices}

\question
Which of these statements seems most appropriate for describing the
sequence whose initial terms are
\(\<1,\frac{3}{4},\frac{5}{8},\frac{9}{16},\frac{17}{32},\dots\>\)?
\begin{choices}
\item The sequence appears to converge to \(\frac{1}{2}\).
\item The sequence appears to diverge to \(\frac{1}{2}\).
\item The sequence appears to neither converge nor diverge.
\end{choices}





\end{questions}

\end{document}
